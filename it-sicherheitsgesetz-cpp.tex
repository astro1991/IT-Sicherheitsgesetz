\documentclass[11pt]{beamer}
\usetheme{default}

\usepackage[utf8]{inputenc}
\usepackage[T1]{fontenc}
\usepackage{graphicx}

\author{Andreas Barchfeld}
\title{Das IT - Sicherheitsgesetz}
%\subtitle{}
%\logo{}
%\institute{}
%\date{16.09.2015}
%\subject{}
%\setbeamercovered{transparent}
%\setbeamertemplate{navigation symbols}{}

\begin{document}
	\maketitle
	
	\begin{frame}
		\frametitle{Aufbau des Gesetzes}
		\begin{itemize} %[<+->]
		\item Artikel 1: Änderung des BSI - Gesetzes
		\item Artikel 2: Änderung des Atomgesetzes
		\item Artikel 3: Änderung des Energiewirtschaftsgesetzes
		\item Artikel 4: Änderung des Telemediengesetzes
		\item Artikel 5: Änderung des Telekommunikationsgesetzes
		\item Artikel 6: Änderung des Bundesbesoldungsgesetzes
		\item Artikel 7: Änderung des Bundeskriminalamtsgesetzes
		\item Artiekl 8: Weitere Änderungen des BSI-Gesetzes
		\item Artikel 9: Änderung des Gesetzes zur Strukturreform des Gebührenrechts des Bundes
		\item Artikel 10: Evaluierung
		\item Artikel 11: Inkrafttreten			
		\end{itemize}
	\end{frame}
	
	\begin{frame}
		\frametitle{Artikel 1, §1}
		\begin{itemize}
			\item "Das Bundesamt ist zuständig für die Informationssicherheit auf nationaler Ebene. Es untersteht dem Bundesministerium des Innern."
			\item "Kritische Infrastrukturen im Sinne dieses Gesetzes sind Einrichtungen, Anlagen oder Teile davon, die
			\begin{enumerate}
				\item den Sektoren ..., Gesundheit, ...
				\item von hoher Bedeutung für das Funktionieren des Gemeinwesens sind, ...
			\end{enumerate}
			\item Die kritischen Infrastrukturen im Sinne dieses Gesetzes werden durch die Rechteverordnung nach §10 Absatz 1 näher bestimmt.
		\end{itemize}
	\end{frame}
	
	\begin{frame}
		\frametitle{Artikel 1, §8a}
		(1) Betreiber kritischer Infrastrukturen sind verpflichtet, spätestens zwei Jahre nach Inkrafttreten der Rechtsverordnung nach §10 Absatz 1 angemessene organisatorische und technische Vorkehrungen zur Vermeidung von Störungen der Verfügbarkeit, Integrität, Authentizität und Vertraulichkeit ihrer informationstechnischen Systeme, Komponenten oder Prozesse zu treffen, die für die Funktionsfähigkeit der von ihnen betriebenen Kritischen Infrastrukturen maßgeblich sind. Dabei soll der Stand der Technik eingehalten werden. ...
	\end{frame}	

	\begin{frame}
		\frametitle{Artikel 1, §8a}
		(3) Die Betreiber Kritischer Infrastrukturen haben mindestens alle zwei Jahre die Erfüllung der Anforderungen nach Absatz 1 auf geeignete Weise nachzuweisen. Der Nachweis kann durch Sicherheitsaudits, Prüfungen oder Zertifizierungen erfolgen. ...
	\end{frame}	

	\begin{frame}
		\frametitle{Artikel 1, §8b}
		(3) Die Betreiber Kritischer Infrastrukturen haben dem Bundesamt binnen sechs Monaten nach Inkrafttreten der Rechtsverordnung nach §10 Absatz 1 eine Kontaktstelle für die Kommunikationsstrukturen nach §3 Absatz 1 Satz 2 Nummer 15 zu benennen. Die Betreiber haben sicherzustellen, dass sie hierüber jederzeit erreichbar sind.
		
		(4) Betreiber Kritischer Infrastrukturen haben erhebliche Störungen der Verfügbarkeit, Integrität, Authentizität und Vertraulichkeit ihrer informationstechnischen Systeme, Komponenten oder Prozesse, die zu einem Ausfall oder einer Beeinträchtigung der Funktionsfähigkeit der von ihnen betriebenen Infrastrukturen  
		
		1. führen können oder
		
		2. geführt haben,
		
		über die Kontaktstelle unverzüglich an das Bundesamt zu melden. ...
	\end{frame}	
	
	\begin{frame}
		\frametitle{Artikel1, §8c}
		Die §§ 8a und 8b sind nicht anzuwenden auf Kleinstunternehmen im Sinne der Empfehlung 2003/361/EC der Kommission vom 6. Mai 2003 betreffend die Definition der Kleinstunternehmen sowie der kleinen und mittleren Unternehmen (ABl. L 124 vom 20.05.2003, S.36). ...
		
		"Ein mittleres Unternehmen wird definiert als ein Unternehmen, das weniger als 250 Mitarbeiter beschäftigt und dessen Umsatz 50 Mio Euro oder dessen Jahresbilanz 43 Mio. Euro nicht überschreitet."
	\end{frame}
	
	\begin{frame}
		\frametitle{Artikel 1, § 14}
		(1) Ordnungswidrig handelt, wer vorsätzlich oder fahrlässig
		
		1. entgegen §8a Absatz 1 Satz 1 in Verbindung mit einer Rechtsverordnung nach §10 Absatz 1 Satz 1 eine dort genannte Vorkehrung nicht, nicht richtig, nicht vollständig oder nicht rechtzeitig trifft, ...
		
		(2) Die Ordnungswidrigkeit kann in den Fällen des Absatzes 1 Nummer 2 Buchstabe b mit einer Geldbuße bis zu hunderttausend Euro, in den übrigen Fällen des Absatzes 1 mit einer Geldbuße bis zu fünfzigtausend Euro geahndet werden.
		
	\end{frame}

	\begin{frame}
		\frametitle{Artikel 4, §13 TMG}
		(7) Diensteanbieter haben, soweit dies technisch möglich und wirtschaftlich zumutbar ist,
		im Rahmen ihrer jeweiligen Verantwortlichkeit für geschäftsmäßig angebotene Telemedien durch
		technische und organisatorische Vorkehrungen sicherzustellen, dass
		
		1. kein unerlaubter Zugriff auf die für ihre Telemedienangebote genutzten technischen Einrichtungen möglich ist und
		
		2. diese
		
		a) gegen Verletzungen des Schutzes personenbezogener Daten und
		
		b) gegen Störungen, auch soweit sie durch äußere Angriffe bedingt sind, gesichert sind. Vorkehrungen nach Satz 1 müssen den Stand der Technik berücksichtigen. Eine Maßnahme nach Satz 1 ist insbesondere die Anwendung eines als sicher anerkannten Verschlüsselungsverfahrens.
	\end{frame}
	
	\begin{frame}
		\frametitle{Artikel 5, §100 TKG}
		(1) Soweit erforderlich, darf der Diensteanbieter die Bestandsdaten und Verkehrsdaten der Teilnehmer und Nutzer erheben und verwenden, um Störungen oder Fehler an Telekommunikationsanlagen zu erkennen, einzugrenzen oder zu beseitigen. Dies gilt
		auch für Störungen, die zu einer Einschränkung der Verfügbarkeit von Informations- und Kommunikationsdiensten oder zu einem unerlaubten Zugriff auf Telekommunikations- und Datenverarbeitungssysteme der Nutzer führen können.
	\end{frame}

	\begin{frame}
		\frametitle{Detailierung aus Begründung}
		\begin{itemize}
			\item Besonders kritische Prozesse bedürfen im Einzelfall besonderer Sicherheitsmaßnahmen durch Abschottung. Diese Prozesse sollten weder mit dem Internet oder öffentlichen Netzen verbunden noch über das Internet angebotenen Diensten abhängig sein.
			\item Stand der Technik in diesem Sinne ist der Entwicklungsstand fortschrittlicher Verfahren, Einrichtungen oder Betriebsweisen, der die praktische Eignung einer Maßnahme zum Schutz ... gesichert erscheinen lässt.
		\end{itemize}
	\end{frame}
	

	\begin{frame}
		\frametitle{geplante Umsetzungskosten, obere Schätzung}
		\begin{tabular}{|c|c|c|c|}
			\hline \rule[-2ex]{0pt}{5.5ex} Amt & Planstellen & Personalkosten p.a. & Sachkosten \\ 
			\hline \rule[-2ex]{0pt}{5.5ex} BSI & 216 & 16 Mio & 7 Mio einmalig \\ 
			\hline \rule[-2ex]{0pt}{5.5ex} BKA & 78 & 5 Mio & 630.000 p.a. \\ 
			\hline \rule[-2ex]{0pt}{5.5ex} BfV & 49 & 3 Mio & 610.000 p.a. \\ 
			\hline \rule[-2ex]{0pt}{5.5ex} BND & 30 & 2 Mio & 688.000 p.a. \\ 
			\hline 
		\end{tabular} 
	\end{frame}

	\begin{frame}
		\frametitle{Quellen}
		\begin{itemize}
			\item Bundesgesetzblatt Jahrgang 2015 Teil I Nr. 31 S. 1324ff
			\item Deutscher Bundestag Drucksache 18/4096
			\item Deutscher Bundestag Drucksache 18/5121
			\item Amtsblatt L124 vom 20.05.2003
		\end{itemize}
	\end{frame}
		
\end{document}
